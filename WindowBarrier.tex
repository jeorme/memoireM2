%%%%%%%%%%%%%%%%%%%%%%%%%%%%%%%%%%%%%%%%%
% Wenneker Assignment
% LaTeX Template
% Version 2.0 (12/1/2019)
%
% This template originates from:
% http://www.LaTeXTemplates.com
%
% Authors:
% Vel (vel@LaTeXTemplates.com)
% Frits Wenneker
%
% License:
% CC BY-NC-SA 3.0 (http://creativecommons.org/licenses/by-nc-sa/3.0/)
% 
%%%%%%%%%%%%%%%%%%%%%%%%%%%%%%%%%%%%%%%%%

%----------------------------------------------------------------------------------------
%	PACKAGES AND OTHER DOCUMENT CONFIGURATIONS
%----------------------------------------------------------------------------------------

\documentclass[11pt]{scrartcl} % Font size
\usepackage{hyperref} % for url
\usepackage{tikz} %for RNN

\input{structure.tex} % Include the file specifying the document structure and custom commands
\newtheorem{theorem}{Theorem}[section]
\newtheorem{corollary}{Corollary}[theorem]
\newtheorem{lemma}[theorem]{Lemma}
\newtheorem{Def}[theorem]{Definition}
\newtheorem{pro}[theorem]{Proposition}

%%newcommand
\newcommand{\Xt}{\left(X_t\right)_{t\in\mathbb{Z}}}
\newcommand{\Z}{\mathbb{Z}}
\newcommand{\C}{\mathbb{C}}
\newcommand{\N}{\mathbb{N}}
\newcommand{\R}{\mathbb{R}}
\newcommand{\var}[1]{\mathbb{V}\left(#1\right)}
\newcommand{\E}[1]{\mathbb{E}\left(#1\right)}
\newcommand{\bb}[1]{\mathcal{BB}\left(0,#1\right)}
\newcommand{\bbn}[1]{\mathcal{BBN}\left(0,#1\right)}
%----------------------------------------------------------------------------------------
%	TITLE SECTION
%----------------------------------------------------------------------------------------

\title{	
	\normalfont\normalsize
	\textsc{CNAM}\\ % Your university, school and/or department name(s)
	\vspace{25pt} % Whitespace
	\rule{\linewidth}{0.5pt}\\ % Thin top horizontal rule
	\vspace{20pt} % Whitespace
	{\huge Series temporelles : ARIMA, LSTM?}\\ % The assignment title
	\vspace{12pt} % Whitespace
	\rule{\linewidth}{2pt}\\ % Thick bottom horizontal rule
	\vspace{12pt} % Whitespace
}

\author{\LARGE Jérôme Petit} % Your name

\date{\normalsize\today} % Today's date (\today) or a custom date

\begin{document}

\maketitle % Print the title

%----------------------------------------------------------------------------------------
%	FIGURE EXAMPLE
%----------------------------------------------------------------------------------------
%
\section{Introduction}
Window barrier options are extension of American barrier options. The difference being that Window barriers are active only for a subsection of the life of the option. The two main window barrier variations are front-window barriers (barriers active from the horizon until a specified date prior to expiry) and rear-window barriers (barriers actives from a specified date after the horizon until expiry).
\section{Front-Window Barrier options}
Front-window barrier options, also called early ending barriers, have a vanilla payoff at expiry plus single (one) or double (two) American barriers that are active from the horizon but cease to be active on the barrier end date and  the window barriers arae either all knock-out or all knock-in.
%%graphique
\subsection{Formula}
\subsection{Call Single barrier early ending}
\begin{eqnarray}
c_A &=& Se^{(b-r)T_2}\left[M(d_1,\eta e_1,\eta\rho)-\left(\frac{H}{S}\right)^{2(\mu+1)}M(f_1,\etae_3,\eta\rho\right]
\end{eqnarray}
\subsection{Call Single barrier early ending}
\section{Rear-Window Barrier options}
\section{Definition}
\bibliographystyle{amsplain}
\bibliography{C:/Users/jerpetit/PycharmProjects/memoireM2/biblio}
\end{document}
